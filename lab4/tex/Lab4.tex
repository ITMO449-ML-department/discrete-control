\documentclass[a4paper, 14pt]{extarticle}

% Текст
\usepackage[utf8]{inputenc} % UTF-8 кодировка
\usepackage[russian]{babel} % Русский язык
\usepackage{indentfirst} % красная строка в первом параграфе в главе
\usepackage{ dsfont }
% Отображение страниц
\usepackage{geometry} % размеры листа и отступов
% \usepackage{mathabx}
\geometry{
	left=30mm,
	top=20mm,
	right=15mm,
	bottom=25mm,
	marginparsep=0mm,
	marginparwidth=0mm,
	headheight=10mm,
	headsep=7mm,
	foot=0mm}
\usepackage{afterpage,fancyhdr} % настройка колонтитулов

\setlength{\baselineskip}{1.5em}
\usepackage{titlesec}
\renewcommand{\thesection}{}
\renewcommand{\thesubsection}{\arabic{subsection}}
\titleformat{\section}[block]{\centering\bfseries\Large}{}{5pt}{}
\titleformat{\subsection}[block]{\bfseries\large}{}{5pt}{}

\pagestyle{fancy}
\fancypagestyle{style}{ % создание нового стиля style
	\fancyhf{} % очистка колонтитулов
    \fancyhead[LO, RE]{\nouppercase{ДСУ}} % название документа наверху
    \fancyhead[RO, LE]{\nouppercase{\leftmark}} % название section наверху
	\fancyfoot[C]{\thepage} % номер страницы справа внизу на нечетных и слева внизу на четных
	\renewcommand{\headrulewidth}{0.25pt} % толщина линии сверху
	\renewcommand{\footrulewidth}{0pt} % толцина линии снизу
}
\fancypagestyle{plain}{ % создание нового стиля plain -- полностью пустого
	\fancyhf{}
	\renewcommand{\headrulewidth}{0pt}
}
\fancypagestyle{title}{ % создание нового стиля title -- для титульной страницы
	\fancyhf{}
	\fancyhead[C]{{\footnotesize
			Министерство образования и науки Российской Федерации\\
			Федеральное государственное автономное образовательное учреждение высшего образования
	}}
	\fancyfoot[C]{{\large 
			Санкт-Петербург\\ 2024
	}}
	\renewcommand{\headrulewidth}{0pt}
}

% Математика
\usepackage{amsmath, amsfonts, amssymb, amsthm} % Набор пакетов для математических текстов
\usepackage{cancel} % зачеркивание для сокращений
% Рисунки и фигуры
\usepackage{graphicx} % вставка рисунков
\usepackage{epstopdf}
\usepackage{wrapfig, subcaption} % вставка фигур, обтекая текст
\usepackage{caption} % для настройки подписей
\captionsetup{figurewithin=none,labelsep=period, font={small,it}} % настройка подписей к рисункам
% Рисование
\usepackage{tikz} % рисование
\usepackage{circuitikz}
\usepackage{pgfplots} % графики
\usepgfplotslibrary{fillbetween}
% Таблицы
\usepackage{multirow} % объединение строк
\usepackage{multicol} % объединение столбцов
% Остальное
\usepackage[unicode, pdftex]{hyperref} % гиперссылки
\usepackage{enumitem} % нормальное оформление списков
\usepackage{float}

\setlist{itemsep=0.15cm,topsep=0.15cm,parsep=1pt} % настройки списков
% Теоремы, леммы, определения...
\theoremstyle{definition}
\newtheorem{Def}{Определение}
\newtheorem*{Axiom}{Аксиома}
\theoremstyle{plain}
\newtheorem{Th}{Теорема}
\newtheorem{Task}{Задание}
\newtheorem{Lem}{Лемма}
\newtheorem{Cor}{Следствие}
\newtheorem{Ex}{Пример}
\theoremstyle{remark}
\newtheorem*{Note}{Замечание}
\newtheorem*{Solution}{Решение}
\newtheorem*{Proof}{Доказательство}
% Свои команды
\newcommand{\comb}[1]{\left[\hspace{-4pt}\begin{array}{l}#1\end{array}\right.\hspace{-5pt} } % совокупность уравнений
\newcommand{\rank}{\mathrm{rank}\;}
% Титульный лист

\usepackage{listings}
\newcommand*{\titlePage}{
	\thispagestyle{title}
	\begingroup
	\begin{center}
		%		{\footnotesize
			%			Министерство образования и науки Российской Федерации\\
			%			Федеральное государственное автономное образовательное учреждение высшего образования
			%		}
		%		
		\vspace*{3ex}
		{\small
			САНКТ-ПЕТЕРБУРГСКИЙ НАЦИОНАЛЬНЫЙ ИССЛЕДОВАТЕЛЬСКИЙ УНИВЕРСИТЕТ ИНФОРМАЦИОННЫХ ТЕХНОЛОГИЙ, МЕХАНИКИ И ОПТИКИ	
		}
		
		\vspace*{2ex}
		
		{\normalsize
			Факультет систем управления и робототехники
		}
		
		\vspace*{15ex}
		
		{
			Отчёт по лабораторной работе №4\\
			<<Дискретные регуляторы с заданными характеристиками переходных процессов>>\\
			по дисциплине\\
			<<Дискретные системы управления>>\\
				% \vspace{2em}
				Вариант 9
			
		}
		
	\end{center}
	\vspace*{10ex}
	\begin{flushright}
		{\large 
			\underline{Выполнили}: студенты потока 1.2 \\
			\begin{flushright}
				\textbf{Дюжев В. Д.}\\
				\textbf{Лалаянц К. А.}\\
			\end{flushright}
		}
		\vspace*{5ex}
		{\large 
			\underline{Преподаватель}:\\ 
			\begin{flushright}
            \textit{Краснов А.Ю.}
			\end{flushright}
		}
	\end{flushright}	
	\newpage
	\setcounter{page}{1}
	\endgroup}
%\usepackage{newtxmath,newtxtext}
%\lstset{literate={а}{\cyra}1{б}{\cyrb}1{в}{\cyrv}1{г}{\cyrg}1{д}{\cyrd}1{е}{\cyre}1{ж}{\cyrzh}1{з}{\cyrz}1{и}{\cyri}1{к}{\cyrk}1{л}{\cyrl}1{м}{\cyrm}1{н}{\cyrn}1{о}{\cyro}1{п}{\cyrp}1{р}{\cyrr}1{с}{\cyrs}1{т}{\cyrt}1{у}{\cyru}1{ф}{\cyrf}1{х}{h}1{ц}{w}1{ч}{\cyrch}1{ш}{\cyrsh}1{щ}{\cyrshch}1{ь}{m}1{ъ}{m}1{ы}{y}1{э}{e}1{ю}{\cyryu}1{я}{\cyrya}}

\lstset{basicstyle=\small}
\newcommand{\tasknum}[3]{Task}%\textunderscore{#1}\textunderscore{#2}y\textunderscore{#2}\textunderscore{#3}}
\usepackage{pdfpages}

\newcommand{\mat}[1]{\begin{pmatrix}#1\end{pmatrix}} 
\newcommand{\bmat}[1]{\begin{bmatrix}#1\end{bmatrix}} 

\newcommand{\code}[2]
{
\begin{minipage}{0.45\textwidth}
    \textbf{Code:}
    #1
\end{minipage}
}

\begin{document}
\renewcommand{\contentsname}{\hfillОГЛАВЛЕНИЕ\hfill} 
\titlePage
\thispagestyle{plain}
\tableofcontents
\pagestyle{style}

\newpage
\setcounter{page}{1}

% \includepdf[pages={33}, scale=1,addtotoc={1, section, 1, {Текст задания}, text}]{./Tasks.pdf}

\section{Введение}
\subsection{Цель работы}
Изучение различных дискретных алгоритмов управления с заданными характеристиками переходных процессов.

\subsection{Данные варианта}
\begin{itemize}
	\item $T$: 0.5
	\item \(a\): 1.8
	\item \(b\): 7.7
	\item \(\zeta\): 0.69
	\item \(\omega_d\): 5
	\item \(K_\nu\): 0.1
\end{itemize}
\newpage

\section{Основная часть}
ОУ имеет вид:
\[G(s) = \frac{e^{-1.8s}}{1 + 7.7s}\]

В дискретном виде c T = 1c ОУ имеет вид:
\[HG(z) = \frac{1}{z^2} \frac{0.02564z + 09615}{z - 0.8782}\]

Для разомкнутой системы, представляющей из себя последовательно: R -- вход; D -- регулятор; HG -- ОУ + ЭНП; Y -- выход. Передаточная функция замкнутой системы имеет вид
\[\frac{Y(z)}{R(z)} = \frac{D(z)HG(z)}{1 + D(z)HG(z)}\]
Пусть \(T_d = \frac{Y(z)}{R(z)}\) -- желаемое поведение замкнутой системы, тогда
\[D(z) = \frac{1}{HG(z)} \frac{T_d(z)}{1 - T_d(z)}\]



\subsection{Апериодический регулятор с T = 1}
Апериодический регулятор – это регулятор, который обеспечивает слежение за ступенчатым входным сигналом, но с задержкой в несколько периодов дискретности, т. е. требуется, чтобы реакция системы
была равна единице для каждого интервала дискретности после приложения
единичного ступенчатого входного воздействия. В таком случае желаемая передаточная функция замкнутой системы будет иметь вид
\[T_d(z) = z^{-k},\]
где \(k \geq 3\). Выберем \(k = 3\)
\[D(z) = \frac{1}{HG(z)} \frac{z^{-3}}{1 - z^{-3}} = \frac{z^6 - 0.8782z^5}{0.02564 z^7 + 0.09615 z^6 - 0.02564 z^4 - 0.09615 z^3}\]

\begin{figure}
	[H]
	\centering
	\includegraphics[width=350pt]{images/task1_y.png}
	\caption{График выхода объекта при использовании апериодического регулятора.}
	\label{fig:task1_y}
\end{figure}
\begin{figure}
	[H]
	\centering
	\includegraphics[width=350pt]{images/task1_u.png}
	\caption{График выхода апериодического регулятора.}
	\label{fig:task1_u}
\end{figure}
Как видно на рисунках \ref{fig:task1_y}-\ref{fig:task1_u}, желаемое поведение объекта совпадает с действительным. 

График входного воздействия возрастает, вероятно из-за собственного числа -3.75 у регулятора вместе с ненулевой матрицей D у объекта. Засчет этого \(u\) возрастает как степенная функция и прокидывается напрямую в выход, что делает систему неустойчивой спустя какое-то время.
% \newpage
\subsection{Регулятор Далина с T = 1}
Регулятор Далина представляет собой модификацию апериодического регулятора и обеспечивает более плавный экспоненциальный отклик, чем у последнего. 
\[T_d(z) = \frac{1}{z^k} \frac{1 - e^{-T/q}}{z - e^{-T/q}}\]
При \(q = 5\), \(k=3\)
\[D(z)= \frac{0.63 z^7 - 0.78 z^6 + 0.20 z^5}{0.025 z^9 + 0.077 z^8 - 0.067 z^7 + 0.013 z^6 - 0.016 z^5 - 0.054 z^4 + 0.022 z^3}\]

\begin{figure}
	[H]
	\centering
	\includegraphics[width=350pt]{images/task2_y.png}
	\caption{График выхода объекта при использовании регулятора Далина.}
	\label{fig:task2_y}
\end{figure}
\begin{figure}
	[H]
	\centering
	\includegraphics[width=350pt]{images/task2_u.png}
	\caption{График выхода регулятора Далина.}
	\label{fig:task2_u}
\end{figure}
Как видно на рисунках \ref{fig:task2_y}-\ref{fig:task2_u}, желаемое поведение объекта совпадает с действительным. 
Причины, по которым управление стремится в бесконечность те же, что и в прошлом случае.

\subsection{Регулятор с заданным расположением полюсов}
\[HG(z) = \frac{0.03z + 0.03*0.75}{z^2 - 1.5z + 0.5}\]
Задача состоит в том, чтобы разработать дискретный регулятор такой, чтобы замкнутая система имела колебательный отклик с декрементом затухания \(\zeta = 0.69\) и частотой колебаний \(\omega_d = 5\). 
Установившаяся ошибка для ступенчатого входа должна быть равна нулю, а установившаяся ошибка для линейно
нарастающего входа должна быть \(K_\nu  = 0.1\) при периоде дискретизации \(T = 0.5c\). 

Исходя из требований к отклику замкнутой системы, определим полюса передаточной функции:
\[z_{1,2} = exp(-\zeta \omega_d T \pm j \omega_d T \sqrt{1 - \zeta^2})\]
Требуемые полюса примут значения:
\[z_{1,2} = -0.0421 \pm 0.1731j \]

Желаемая передаточная функция замкнутой системы будет иметь вид
\[T(z) = \frac{b_0 + b_1 z^{-1} + b_2 z^{-2} + \dots}{(z^{-1} + 0.0421 + 0.1731j)(z^{-1}  + 0.0421 - 0.1731j)}\]
Из соображений физической реализуемости, \(b_0 = 0\) и лишь коэффициенты \(b_1\) и \(b_1\) будут ненулевыми
\[T(z) = \frac{b_1 z + b_2}{z^2 + 0.0842z + 0.03173602}\]

Установившаяся ошибка примет нулевое значение при \(T(1) = 1\).
\[b_1 + b_2 = 1.11593602\]

Пусть \(K_\nu  = 0.1\)
\[
\frac{dT}{dt} \Bigg|_{z=1} = \frac{b_1 ( z^2 + 0.0842z + 0.03173602) - ( b_1 z + b_2 ) ( 2z + 0.0842 )}{(  z^2 + 0.0842z + 0.03173602)^2} = -\frac{1}{K_v T} = -20.
\]
Итого получается система:
\[
\begin{cases}
	b_1 + b_2 = 1.11593602 \\
	2.56974391511729b_1 +1.67363519375888b_2 = -20
\end{cases}
\]  
\[
\begin{cases}
	b_1 = -24.4029204000002 \\
	b_2 = 25.5188564200002
\end{cases}
\]
\[C(z) = \frac{-24.4 z^5 + 60.07 z^4 - 46.02 z^3 + 10.48 z^2 - 0.5277 z + 0.4049}{0.03 z^5 + 0.7596 z^4 - 0.149 z^3 - 0.5674 z^2 - 0.05507 z - 0.0182}\]
Как видно на рисунках \ref{fig:task3_y}-\ref{fig:task3_u}, желаемое поведение объекта совпадает с действительным. 

\begin{figure}
	[H]
	\centering
	\includegraphics[width=350pt]{images/task3_y.png}
	\caption{График выхода объекта при использовании регулятора с заданным расположением полюсов.}
	\label{fig:task3_y}
\end{figure}
\begin{figure}
	[H]
	\centering
	\includegraphics[width=350pt]{images/task3_u.png}
	\caption{График выхода регулятора с заданным расположением полюсов.}
	\label{fig:task3_u}
\end{figure}



\newpage
\section{Выводы}
В ходе выполнения работы ознакомились с методами синтеза классических регуляторов для дискретных систем. Теоретические выкладки, сделанные в соответствующей секции были подтверждены во время про ведения экспериментов, что можно наблюдать на графиках моделирования.


\end{document}