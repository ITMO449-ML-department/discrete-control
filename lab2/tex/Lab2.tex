\documentclass[a4paper, 14pt]{extarticle}

% Текст
\usepackage[utf8]{inputenc} % UTF-8 кодировка
\usepackage[russian]{babel} % Русский язык
\usepackage{indentfirst} % красная строка в первом параграфе в главе
\usepackage{ dsfont }
% Отображение страниц
\usepackage{geometry} % размеры листа и отступов
% \usepackage{mathabx}
\geometry{
	left=30mm,
	top=20mm,
	right=15mm,
	bottom=25mm,
	marginparsep=0mm,
	marginparwidth=0mm,
	headheight=10mm,
	headsep=7mm,
	foot=0mm}
\usepackage{afterpage,fancyhdr} % настройка колонтитулов

\setlength{\baselineskip}{1.5em}
\usepackage{titlesec}
\renewcommand{\thesection}{}
\renewcommand{\thesubsection}{\arabic{subsection}}
\titleformat{\section}[block]{\centering\bfseries\Large}{}{5pt}{}
\titleformat{\subsection}[block]{\bfseries\large}{}{5pt}{}

\pagestyle{fancy}
\fancypagestyle{style}{ % создание нового стиля style
	\fancyhf{} % очистка колонтитулов
    \fancyhead[LO, RE]{\nouppercase{ДСУ}} % название документа наверху
    \fancyhead[RO, LE]{\nouppercase{\leftmark}} % название section наверху
	\fancyfoot[C]{\thepage} % номер страницы справа внизу на нечетных и слева внизу на четных
	\renewcommand{\headrulewidth}{0.25pt} % толщина линии сверху
	\renewcommand{\footrulewidth}{0pt} % толцина линии снизу
}
\fancypagestyle{plain}{ % создание нового стиля plain -- полностью пустого
	\fancyhf{}
	\renewcommand{\headrulewidth}{0pt}
}
\fancypagestyle{title}{ % создание нового стиля title -- для титульной страницы
	\fancyhf{}
	\fancyhead[C]{{\footnotesize
			Министерство образования и науки Российской Федерации\\
			Федеральное государственное автономное образовательное учреждение высшего образования
	}}
	\fancyfoot[C]{{\large 
			Санкт-Петербург\\ 2024
	}}
	\renewcommand{\headrulewidth}{0pt}
}

% Математика
\usepackage{amsmath, amsfonts, amssymb, amsthm} % Набор пакетов для математических текстов
\usepackage{cancel} % зачеркивание для сокращений
% Рисунки и фигуры
\usepackage{graphicx} % вставка рисунков
\usepackage{epstopdf}
\usepackage{wrapfig, subcaption} % вставка фигур, обтекая текст
\usepackage{caption} % для настройки подписей
\captionsetup{figurewithin=none,labelsep=period, font={small,it}} % настройка подписей к рисункам
% Рисование
\usepackage{tikz} % рисование
\usepackage{circuitikz}
\usepackage{pgfplots} % графики
\usepgfplotslibrary{fillbetween}
% Таблицы
\usepackage{multirow} % объединение строк
\usepackage{multicol} % объединение столбцов
% Остальное
\usepackage[unicode, pdftex]{hyperref} % гиперссылки
\usepackage{enumitem} % нормальное оформление списков
\usepackage{float}

\setlist{itemsep=0.15cm,topsep=0.15cm,parsep=1pt} % настройки списков
% Теоремы, леммы, определения...
\theoremstyle{definition}
\newtheorem{Def}{Определение}
\newtheorem*{Axiom}{Аксиома}
\theoremstyle{plain}
\newtheorem{Th}{Теорема}
\newtheorem{Task}{Задание}
\newtheorem{Lem}{Лемма}
\newtheorem{Cor}{Следствие}
\newtheorem{Ex}{Пример}
\theoremstyle{remark}
\newtheorem*{Note}{Замечание}
\newtheorem*{Solution}{Решение}
\newtheorem*{Proof}{Доказательство}
% Свои команды
\newcommand{\comb}[1]{\left[\hspace{-4pt}\begin{array}{l}#1\end{array}\right.\hspace{-5pt} } % совокупность уравнений
\newcommand{\rank}{\mathrm{rank}\;}
% Титульный лист

\usepackage{listings}
\newcommand*{\titlePage}{
	\thispagestyle{title}
	\begingroup
	\begin{center}
		%		{\footnotesize
			%			Министерство образования и науки Российской Федерации\\
			%			Федеральное государственное автономное образовательное учреждение высшего образования
			%		}
		%		
		\vspace*{3ex}
		{\small
			САНКТ-ПЕТЕРБУРГСКИЙ НАЦИОНАЛЬНЫЙ ИССЛЕДОВАТЕЛЬСКИЙ УНИВЕРСИТЕТ ИНФОРМАЦИОННЫХ ТЕХНОЛОГИЙ, МЕХАНИКИ И ОПТИКИ	
		}
		
		\vspace*{2ex}
		
		{\normalsize
			Факультет систем управления и робототехники
		}
		
		\vspace*{15ex}
		
		{
			Отчёт по лабораторной работе №2\\
			<<Классические регуляторы для дискретных систем>>\\
			по дисциплине\\
			<<Дискретные системы управления>>\\
				% \vspace{2em}
				Вариант 9
			
		}
		
	\end{center}
	\vspace*{10ex}
	\begin{flushright}
		{\large 
			\underline{Выполнили}: студенты потока 1.2 \\
			\begin{flushright}
				\textbf{Дюжев В. Д.}\\
				\textbf{Лалаянц К. А.}\\
			\end{flushright}
		}
		\vspace*{5ex}
		{\large 
			\underline{Преподаватель}:\\ 
			\begin{flushright}
            \textit{Краснов А.Ю.}
			\end{flushright}
		}
	\end{flushright}	
	\newpage
	\setcounter{page}{1}
	\endgroup}
%\usepackage{newtxmath,newtxtext}
%\lstset{literate={а}{\cyra}1{б}{\cyrb}1{в}{\cyrv}1{г}{\cyrg}1{д}{\cyrd}1{е}{\cyre}1{ж}{\cyrzh}1{з}{\cyrz}1{и}{\cyri}1{к}{\cyrk}1{л}{\cyrl}1{м}{\cyrm}1{н}{\cyrn}1{о}{\cyro}1{п}{\cyrp}1{р}{\cyrr}1{с}{\cyrs}1{т}{\cyrt}1{у}{\cyru}1{ф}{\cyrf}1{х}{h}1{ц}{w}1{ч}{\cyrch}1{ш}{\cyrsh}1{щ}{\cyrshch}1{ь}{m}1{ъ}{m}1{ы}{y}1{э}{e}1{ю}{\cyryu}1{я}{\cyrya}}

\lstset{basicstyle=\small}
\newcommand{\tasknum}[3]{Task}%\textunderscore{#1}\textunderscore{#2}y\textunderscore{#2}\textunderscore{#3}}
\usepackage{pdfpages}

\newcommand{\mat}[1]{\begin{pmatrix}#1\end{pmatrix}} 
\newcommand{\bmat}[1]{\begin{bmatrix}#1\end{bmatrix}} 

\newcommand{\code}[2]
{
\begin{minipage}{0.45\textwidth}
    \textbf{Code:}
    #1
\end{minipage}
}

\begin{document}
\renewcommand{\contentsname}{\hfillОГЛАВЛЕНИЕ\hfill} 
\titlePage
\thispagestyle{plain}
\tableofcontents
\pagestyle{style}

\newpage
\setcounter{page}{1}

% \includepdf[pages={33}, scale=1,addtotoc={1, section, 1, {Текст задания}, text}]{./Tasks.pdf}

\section{Введение}
\subsection{Цель работы}
Ознакомиться с методами синтеза классических регуляторов для дискретных систем.

\subsection{Данные варианта}
\begin{itemize}
	\item Тип ОУ: 2 (рис. \ref{fig:plant_type});
	\begin{figure}
	    [H]
	    \centering
	    \includegraphics[width=350pt]{images/plant_type.png}
	    \caption{Тип ОУ}
		\label{fig:plant_type}
	\end{figure}
	\item $k_1$: 7.29
    \item $a^1_0$: 1
    \item $T_1$: 0.38
    \item $\xi$: 0
    \item $k_2$: 9.46
    \item $a_0^2$: 1
    \item $T_2$: 1
    \item $T$: 0.25
    \item $g(k) = 7.68 \sin{(0.3kT)}$
\end{itemize}
\newpage

\section{Основная часть}
\subsection{Проектирование дискретных стабилизирующих регуляторов}
Дискретный вариант имеет вид:
\[
	A = \begin{bmatrix} 0.3581 & -0.2104 \\ 0.3198 & 0.9387 \end{bmatrix}, \quad
	B = \begin{bmatrix} 1.279 \\ 0.3728 \end{bmatrix}, \quad
	C = \begin{bmatrix} 0 & 11.34 \end{bmatrix}, \quad
	D = \begin{bmatrix} 0 \end{bmatrix}
\]
и имеет cобственные числа 0.5179 и 0.7788.

В качестве эталонной модели возьмем оптимальную по быстродействию дискретную систему, т.е. \(z_i^* = 0,~i=\overline{1,n}\).
Тогда введем эталонную модель:
\[
	\begin{cases}
		\eta(m+1) = \Gamma \eta(m) \\
		\nu(m) = -H \eta(m)
	\end{cases}
\]
\[
	\Gamma = \begin{bmatrix} 0 & 1 \\ 0 & 0 \end{bmatrix}, \quad
	H = \begin{bmatrix} 1 & 1 \end{bmatrix}, \quad
\]
Для синтеза модального регулятора необходимо соблюдение следующих условий:
\begin{itemize}
	\item \(rank(obsv(H, \Gamma)) = n\);
	\item \(rank(ctrb(A, B)) = n\);
	\item у А и \(\Gamma\) нет общих собственных чисел.
\end{itemize}

Известно, что \(F = A - BK\) и \(\Gamma\) подобны. Следовательно \(F = M \Gamma M^{-1}\). Для получения дополнительного матричного уравнения потребуем, чтобы выход эталонной модели совпадал бы с желаемым управляемым воздействием.
Из этого:
\[
	\begin{cases}
		AM - M \Gamma = BH \\
		K = HM^{-1}
	\end{cases} \rightarrow
	\begin{cases}
		K = [0.6122~~1.3780]\\
		A - BK = \begin{bmatrix}
			-0.4250 & -1.9729 \\
			0.0915 & 0.4250
			\end{bmatrix} \\
		\sigma(A - BK) = [-0.54 ~ 0.54]*10^{-8}
	\end{cases} 
\]
Видно, что из-за неточности численного решения уравнения Сильвестра в Matlab, собственные числа незначительно отличаются от 0. Это повышает время сходимости, которое должно было быть равным одному периоду дискретизации благодаря нулевым собственным числам. Результаты моделирования представлены на рис. \ref{fig:task1}.
\begin{figure}
	[H]
	\centering
	\includegraphics[width=350pt]{images/task1.png}
	\caption{График состояния объекта при использовании модального регулятора.}
	\label{fig:task1}
\end{figure}

\subsection{Проектирование дискретных следящих регуляторов}
\subsubsection{Построение задающего воздействия}
Известно, что задающее воздействие имеет вид:
\[
\begin{cases}
	\xi(m+1) = \Gamma \xi(m)\\
	g(m) = H_\xi \xi(m)
\end{cases}
\]
Для построения задающего воздействия были выбраны матрицы непрерывной системы:
\[
	\Gamma_c = \begin{bmatrix} 0 & 0.3 \\ -0.3 & 0 \end{bmatrix}, \quad
	H_{\xi c} = \begin{bmatrix} 7.68 & 0 \end{bmatrix}, \quad
	\xi(0) = \begin{bmatrix} 0 \\ 1 \end{bmatrix};
\]
После дискретизации был получен их дискретный вариант:
\[
    \Gamma = \begin{bmatrix} 0.9972 & 0.07493 \\ -0.07493 & 0.9972 \end{bmatrix}, \quad
    H_\xi = \begin{bmatrix} 7.68 & 0 \end{bmatrix}, \quad
	\xi(0) = \begin{bmatrix} 0 \\ 1 \end{bmatrix};
\]
Сравнение линейного и непрерывного генераторов представлен на рис. \ref{fig:task2_gen}.
\begin{figure}
	[H]
	\centering
	\includegraphics[width=350pt]{images/task2_generator.png}
	\caption{График сравнения дискретного и непрерывного генераторов}
	\label{fig:task2_gen}
\end{figure}
\subsubsection{Синтез регулятора со встроенной моделью}
В дискретном случае:
\[g(m) = H_\xi \Gamma^m \xi(0)\]

Представим объект управления в канонической наблюдаемой форме:
\[
    A = \begin{bmatrix} 0 & 1 \\ -0.4034 & 1.297 \end{bmatrix}, \quad
    B = \begin{bmatrix} 4.229 \\ 8.608 \end{bmatrix}, \quad
    C = \begin{bmatrix} 1 & 0 \end{bmatrix}, \quad
    D = \begin{bmatrix} 0 \end{bmatrix};
\]
При условии, что (A, B) -- полностью управляема, а (С, А) -- полностью наблюдаема, синтезируем регулятор:
\[
\begin{cases}
	e = g - y\\
	\eta(m+1) = \Gamma \eta(m) + B_\eta e() \\
	u(m) = k_1 e(m) + K_\eta \eta(m) - k_2 x_2(m) - \dots - k_n x_n(m)
\end{cases}
\]
\(B_\eta\) выбирается из условия полной управляемости пары \((\Gamma, B_\eta)\). Сведем задачу к модальному управлению:
\[u(m) = k_1 g(m) + K_\eta \eta(m) - K x(m) = k\]

Введем в рассмотрение расширенную модель ОУ, объединив уравнение движения объекта с уравнением встроенной модели:

\[
\bar{x}(m + 1) = 
\begin{bmatrix}
\eta(m + 1) \\ x(m + 1)
\end{bmatrix}
= 
\begin{bmatrix}
\Gamma & -B_\eta C \\
0 & A 
\end{bmatrix}
\begin{bmatrix}
\eta(m) \\ x(m)
\end{bmatrix}
+ 
\begin{bmatrix}
0 \\ B 
\end{bmatrix} u(m)
+ 
\begin{bmatrix}
B_\eta \\ 0 
\end{bmatrix} g(m)
\]

Тогда, с учетом обозначений:
\[
\bar{A} = \begin{bmatrix} \Gamma & -B_\eta C \\ 0 & A \end{bmatrix}, \quad
\bar{B} = \begin{bmatrix} 0 \\ B \end{bmatrix}, \quad
\bar{B}_1 = \begin{bmatrix} B_\eta \\ 0 \end{bmatrix}, \quad
\bar{K} = \begin{bmatrix} -K_\eta & K \end{bmatrix}
\]

получим в итоге для замкнутой системы:
\[
\bar{x}(m + 1) = (\bar{A} - \bar{B} \bar{K}) \bar{x}(m) + (\bar{B}_1 + k_1 \bar{B}) g(m) = \bar{F} x(m) + \bar{B}_g g(m),
\]

где \(\bar{F} = \bar{A} - \bar{B} \bar{K}\) — матрица размерности \((n + q) \times (n + q)\), определяющая динамические свойства замкнутой системы; \(\bar{B}_g = \bar{B}_1 + k_1 \bar{B}\) — матрица входов по задающему воздействию размерности \((n + q) \times 1\).

Если пара \(A, B\) полностью управляема, то выбором матрицы \(\bar{K}\) можно обеспечить произвольные желаемые корни характеристического полинома или коэффициенты уравнения замкнутой системы. Отметим, что пара \(\bar{A}, \bar{B}\) будет полностью управляема при выполнении следующих условий:

\begin{itemize}
    \item пара \(A, B\) полностью управляема, а пара \(C, A\) полностью наблюдаема;
    \item пара \(\Gamma, B_\eta\) полностью управляема.
\end{itemize}

Итого получаем
\[
\bar A =
	\begin{bmatrix}
	0.9972 & 0.0749 & -1.0000 & 0 \\
	-0.0749 & 0.9972 & -1.0000 & 0 \\
	0 & 0 & 0 & 1.0000 \\
	0 & 0 & -0.4034 & 1.2967
	\end{bmatrix};
\bar B = \begin{bmatrix}
	0 \\
	0 \\
	4.2286 \\
	8.6085
	\end{bmatrix};
\bar K_1^T = \begin{bmatrix}
	-0.3568 \\ 0.1592 \\ 0.1383 \\ 0.1982
	\end{bmatrix};
\]
\[\sigma(\bar{A} - \bar{B} \bar{K_1}) = [0.1,~0.2,~0.3,~0.4]\]

Как видно на рис. \ref{fig:task2_y} задача слежения выполняется, ошибка сводится к 0.
\begin{figure}
	[H]
	\centering
	\includegraphics[width=350pt]{images/task2_system.png}
	\caption{График слежения дискретной системы с ПВВ за генератором}
	\label{fig:task2_y}
\end{figure}


\subsection{Построение регуляторов для объектов с неполной информацией}
\subsubsection{Устройства оценки полной размерности}
\[
\begin{cases}
	x(m+1) = A x(m) + B u(m) \\
	y_m(m) = C_m x(m)
\end{cases}
\]
\(y_m \in \mathds{R}^l\) -- вектор измеряемых величин, \(l < n\). 
Будем строить оценку \(\hat x_m\)
\[\hat x(m+1) = A \hat x(m) + B u(m) + L(y_m(m) - C_m \hat x(m))\]
% \[\hat x(m+1) = A \hat x(m) + B u(m) + L(y_m(m) - C_m \hat x(m))\]

Задача заключается в нахождении такой матрицы \(L\), чтобы вектор невязки свелся к 0. Рассмотрим
\[\hat x(m+1) - x(m+1) = A \hat x(m) + B u(m) + L(y_m(m) - C_m \hat x(m)) - A x(m) + B u(m)\]
\[\hat x(m+1) - x(m+1) = (A - LC_m)(\hat x(m+1) - x(m+1))\]
Введем обозначение вектора невязки в виде:
\[
\begin{cases}
	\tilde x(m) = \hat x(m) - x(m) \\
	F_e = A - LC_m
\end{cases} \rightarrow
	\tilde x(m+1) = F_e  \tilde x(m)
\]
Для нахождения \(L\) задается матрица \(\Gamma_0 \in \mathds{R}^{n \times n}\) с желаемыми собственными числами и  \(H \in \mathds{R}^{l \times n}\), такая что \((H, \Gamma_0)\) полностью наблюдаема.
\[
	\begin{cases}
		A^T M - M \Gamma_0 = C^T H \\
		L = H M^{-1} \\
	\end{cases}
\]

\subsubsection{Проектирование динамического регулятора с устройством оценки полной размерности}

Динамическим регулятором будем называть совокупность устройства оценки и регулятора, реализующего соотношение
\[u(m) = -K \hat x(m)\]
Тогда
\[
% \begin{cases}
	x(m+1) = A x(m) - B K \hat x(m) = A x(m) - BK (x + \tilde x) = Fx(m) - BK \tilde x
% \end{cases}
\]

Откуда
\[
\begin{cases}
	x(m+1) = A x(m) + B u(m) \\
	\tilde x(m+1) = F_e  \tilde x(m) \\
\end{cases} \rightarrow
\begin{bmatrix}
x(m+1) \\
\tilde x(m+1)
\end{bmatrix}
=
\begin{bmatrix}
	F & -BK \\
	0 & F_e
\end{bmatrix}
\begin{bmatrix}
	x(m) \\
	\tilde x(m)
\end{bmatrix}
\]
\[\bar F = \begin{bmatrix}
	F & -BK \\
	0 & F_e
\end{bmatrix}\]
Из верхнетреугольного вида матрицы \(\bar F\) следует, что свойства замкнутой
системы зависят от свойств матрицы \(F = A - BK\) и свойств матрицы \(Fe = A - LC\) и определитель матрицы замкнутой системы равен произведению определителей матриц \(F\) и \(F_e\), а корни характеристического уравнения объединяют корни характеристических уравнений матриц \(F\) и \(F_e\).

Из полученного следует свойство разделения, которое заключается в следующем:
\begin{enumerate}
	\item \(n\) желаемых корней замкнутой системы можно обеспечить выбором матрицы обратных связей K, т. е. если A, B -- полностью управляемая пара, то существует матрица K такая, что n собственных чисел матрицы A - BK равны \(\{z_1^* \dots z_n^*\}\);
	\item оставшиеся n желаемых корней можно обеспечить выбором матрицы входов устройства оценки \(C_m\), т. е. если \(C_m\), A -- полностью наблюдаемая пара, то существует матрица L такая, что n желаемых собственных чисел матрицы \(F_e =A-LC_m\) и равны \(\{z_{e1}^* \dots z_{e1}^*\}\).
\end{enumerate}

\subsubsection{Результаты для модального регулятора}
Были использованы полученная ранее матрица \(K_1\) и
\(
L = \begin{bmatrix}
	0.0168 & 0.1143
\end{bmatrix};
\). На рис. \ref{fig:task3_1_y} -\ref{fig:task3_1_e} видно, что ошибки свелись к 0.
\begin{figure}
	[H]
	\centering
	\includegraphics[width=0.6\textwidth]{images/task3_1_y.png}
	\caption{График выхода дискретной системы с неполной информацией и модальным регулятором}
	\label{fig:task3_1_y}
\end{figure}

\begin{figure}
	[H]
	\centering
	\includegraphics[width=0.6\textwidth]{images/task3_1_e.png}
	\caption{График вектор невязки наблюдателя дискретной системы с неполной информацией и модальным регулятором}
	\label{fig:task3_1_e}
\end{figure}

\subsubsection{Результаты для МВВ}
\begin{figure}
	[H]
	\centering
	\includegraphics[width=0.6\textwidth]{images/task3_2_y.png}
	\caption{График выхода дискретной системы с неполной информацией и модальным регулятором}
	\label{fig:task3_2_y}
\end{figure}

\begin{figure}
	[H]
	\centering
	\includegraphics[width=0.6\textwidth]{images/task3_2_e.png}
	\caption{График вектор невязки наблюдателя дискретной системы с неполной информацией и модальным регулятором}
	\label{fig:task3_2_e}
\end{figure}

\begin{figure}
	[H]
	\centering
	\includegraphics[width=0.6\textwidth]{images/task3_2_x.png}
	\caption{График вектор состояния наблюдателя дискретной системы с неполной информацией и модальным регулятором}
	\label{fig:task3_2_x}
\end{figure}




\newpage
\section{Выводы}
В ходе выполнения работы ознакомились с методами синтеза классических регуляторов для дискретных систем. Теоретические выкладки, сделанные в соответствующей секции были подтверждены во время про ведения экспериментов, что можно наблюдать на графиках моделирования.


\end{document}